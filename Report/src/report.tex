%@descr: A template for a paper, a report or a thesis - suitable for modifications
%@author: Maciej Komosiński

\documentclass{article} 
\usepackage[english]{babel} 
\usepackage[utf8]{inputenc} 
\usepackage[OT4]{fontenc}
\usepackage{graphicx,color} %include pdf's (and png's for raster graphics... avoid raster graphics!)
\usepackage{amsmath} %just for \eqref
\usepackage{url}
\usepackage{subcaption} % or \usepackage{subfigure}
\usepackage{float} % for [H] placement specifier
\usepackage{multicol}
\usepackage[pdftex,hyperfootnotes=false,pdfborder={0 0 0}]{hyperref} % after all packages; pdfborder not implemented the same way for every implementation because the specification is imprecise; under miktex you just don't see the frames


% Changing the size of the text page
\addtolength{\voffset}{-2cm}
\addtolength{\hoffset}{-2cm}
\addtolength{\textwidth}{4cm}
\addtolength{\textheight}{4cm}

% More reasonable parameters controlling the placement of figures
\renewcommand{\topfraction}{.85}
\renewcommand{\bottomfraction}{.7}
\renewcommand{\textfraction}{.15}
\renewcommand{\floatpagefraction}{.66}
\renewcommand{\dbltopfraction}{.66}
\renewcommand{\dblfloatpagefraction}{.66}
\setcounter{topnumber}{9}
\setcounter{bottomnumber}{9}
\setcounter{totalnumber}{20}
\setcounter{dbltopnumber}{9}

% Custom bullet list with small vertical spacing
\newenvironment{tightlist}{
\begin{itemize}
  \setlength{\itemsep}{1pt}
  \setlength{\parskip}{0pt}
  \setlength{\parsep}{0pt}}
{\end{itemize}}

% We are looking for images in the following directory:
\graphicspath{{pics/}}



%\title{}
%\author{}
%\date{}


\begin{document}

\thispagestyle{empty} %no page number

\begin{center}
{\large{Machine Learning Theory:\\
Miniproject\\
}}

\vspace{3ex}

Computational experiment: Kolmogorov-Arnold Networks

\vspace{3ex}
{\footnotesize\today}

\end{center}


\vspace{10ex}

Lecturer: dr hab.~inż. Wojciech Kotłowski, prof. PP

\vspace{5ex}

Authors:
\begin{tabular}{lllr}
\textbf{Jędrzej Warczyński} & inf148234 & AI & jedrzej.warczynski@student.put.poznan.pl \\
\textbf{Mateusz Zwierzchowski} & inf148xxx & AI & mateusz.zwierzchowski@student.put.poznan.pl \\
\end{tabular}

\vspace{25ex}

\noindent We declare that this report and the accompanying source code has been prepared solely by the above author(s), and all contributions from other sources have been properly indicated and are cited in the bibliography.

\newpage






\begin{abstract}
Kolmogorov-Arnold Networks (KANs), proposed as a promising alternative to Multilayer Perceptrons (MLPs),
employ learnable activation functions on edges, represented as spline functions, inspired by the Kolmogorov-Arnold representation theorem.
This paper investigates the influence of spline order and grid size on training, validation, and test accuracy, as well as loss, in KAN networks.
Experimental analyses are conducted on two datasets, MNIST and CIFAR10, to explore the performance of KANs under varying parameters.
\end{abstract}

\section{Introduction}\label{sec:introduction}

KANs\cite{liu2024kan}, introduced as a promising alternative to Multilayer Perceptrons (MLPs),
are inspired by the Kolmogorov-Arnold representation theorem rather than the universal approximation theorem that underpins MLPs.
While MLPs utilize fixed activation functions on nodes, KANs employ learnable activation functions on edges, represented as spline functions.
This unique architecture eliminates traditional linear weight matrices in favor of a flexible and adaptive framework, potentially offering advantages in accuracy and parameter efficiency.

The focus of our research lies in understanding how spline order and grid size impact the training, validation, testing accuracy, and loss in KANs.
Through extensive empirical experiments conducted on two datasets, MNIST and CIFAR10, we aim to elucidate the optimal configuration of KANs for various tasks and datasets.

\section{Experiments}\label{sec:experiments}

\subsection{Experimental Setup}\label{subsec:experimental-setup}

We conduct experiments on the MNIST and CIFAR10 datasets to evaluate the performance of KANs with varying spline orders and grid sizes.
We set the number of layers to 4.
The number of nodes in the hidden layers are 128, 64, 32, and 10, respectively.
We use the Adam optimizer with a learning rate of 0.001 with exponential decay of 0.8 and a batch size of 64.
The training process is run for 10 epochs.
We evaluate the performance of KANs based on training, validation, and test accuracy, as well as loss.
We have trained KANs with fixed spline order of 3 and grid sizes of 5, 10, 20, 40, 80, 160, 320, 640, and 1280 as well as
models with fixed size of gird equal to 5 and spline orders ranging from 2 to 12.


\subsection{MNIST Dataset}\label{subsec:mnist}

The MNIST dataset consists of 60,000 training and 10,000 testing grayscale images of handwritten digits.
The comaprison of traing and validation loss is presented on the left and right side of Figure~\ref{fig:pics/mnist_loss} respectively.
The comparison of training, validation, and test accuracy is presented in Figure~\ref{fig:pics/mnist_accuracy}.


\subsection{CIFAR10 Dataset}\label{subsec:cifar10}

The CIFAR10 dataset comprises 50,000 training and 10,000 testing color images of 10 classes.
Similarly to the MNIST dataset, we present the comparison of training and validation loss in Figure~\ref{fig:pics/cifar10_loss}
and the comparison of training, validation, and test accuracy in Figure~\ref{fig:pics/cifar10_accuracy}.


\section{Conclusions}\label{sec:conclusions}


\clearpage % let LaTeX put pending figures right here -- this command "releases" the accumulated content, which is useful if you have placed a lot of images, much more than text -- so they do not appear at the end of the document.


%%%%%%%%%%%%%%%% references %%%%%%%%%%%%%%%%

\bibliography{biblio}
\bibliographystyle{plain}


\end{document}
